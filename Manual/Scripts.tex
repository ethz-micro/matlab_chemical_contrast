%!TEX root = Chemical_contrast.tex
\section{Scripts}

The following scripts will be described:

\begin{enumerate}
	\item Code2CreateSfemBorder.m;
	\item Code2CreateStmBorder.m;
	\item Code2GetContrast.m;
	\item Code2GetProfileLine.m;
	\item Code2GetResolution.m;
	\end{enumerate}

All of these scripts are structured in the same way. First is asked to the user to add the path of the NanoLib library, the path of the Functions folder, and to choose which file has to be open.
Following, the file is loaded using the function loadProcessedSxM of the NanoLib library(see NanoLib user guide section 2.2.3). The index i of the channel desired is given by the function getChannel(see NanoLib user guide section 4.0.4) and so the data of that channel are easily found in file.channels(i).data . At this point one of the function in the folder Functions is called using as input the data of the channel chosen.

\subsection{Code2CreateSfemBorder}

This script opens a sxm file and applies \textbf{CreateStepBorder} function either to a Current map or to a Secondary electron map(''Current'' or ''Channeltron'' channels).

\subsubsection{Function CreateStepBorder}

This function acts as following:
\begin{itemize}  
\item Plots alldata matrix using the colormap defined by nanonisMap (NanoLib/op/nanonisMap.m).\\
\textbf{Note:}The x and y axis are in pixels and the origin is top/left corner(matrix representation).
\item Allows the user to select the region of interest(roi) inside the image loaded by clicking 
and dragging the mouse. When double click the fit starts.
\item Fits to every scan line inside the roi the step function \textit{$-A*tanh(B*(x-C)-D$} with $A,B,C,D>0$; 
the fit parameter $C$ is $x$ coordinate of the inflection point of the tanh function
i.e. the position of the step in a scanline. The $y$ coordinate is the scan line index.
All of these $x,y$ coordinates define the points of the border line between W and C regions.\\
\textbf{Note:} The shape of the function used for the fit defines an up/down step from left to right.
\item Asks to the user whether to save the border points in a txt file.
\item Plots the border line points as a  green line.
\item Creates a figure where only the roi is visible.
\end{itemize}

\subsection{Code2CreateStmBorder}

This script opens a sxm file and applies \textbf{CreateFreeHandBorder} function to a topography map(''Z'' channel).

\subsubsection{Function CreateFreeHandBorder}

This function acts as following:
\begin{itemize}  
\item Plots alldata matrix using the colormap defined by nanonisMap (NanoLib/op/nanonisMap.m).\\
\textbf{Note:}The x and y axis are in pixels and the origin is top/left corner(matrix representation).
\item Allows the user to draw a border inside the image loaded by clicking and dragging the mouse.
\item When double click on the line, the $x$ and $y$ coordinates of those points are saved in the txt file typed.
\end{itemize}

\subsection{Code2GetContrast}

This script opens a sxm file and applies \textbf{GetContrast} function either to a Current map or to a Secondary electron map(Current or Channeltron channels).

\subsubsection{Function GetContrast}

This function acts as following:
\begin{itemize}  
\item Plots alldata matrix using the colormap defined by nanonisMap (NanoLib/op/nanonisMap.m).\\
\textbf{Note:}The $x$ and $y$ axis are in pixels and the origin is top/left corner(matrix representation).
\item Lets the user to draw a first region A inside the image loaded by clicking 
  and dragging the mouse.
\item Lets the user to draw a second region B inside the image loaded by clicking 
  and dragging the mouse.
\item Plots the histogram of the values for the two regions. A normal distribution might justify a total contrast calculation. 
\item Plots the histogram of the contrast for every line. If the line doesn't contain values for one of the two regions, then that scan line is not counted for the histogram and the contrast calculation.
\item The contrast per line and the total contrast is calculated as following: Contrast $= |(A - B)/(A + B)|$ where A and B are the mean values of the regions.
\item The mean values of the two regions and the total contrast are shown in the title of the ''Contrast per line'' figure.
\end{itemize}

\subsection{Code2GetProfileLine}

This script opens a sxm file and applies \textbf{GetProfileLine} function to a topography map(''Z'' channel).

\subsubsection{Function GetProfileLine}

This function acts as following:
\begin{itemize}  
\item Plots alldata matrix using the colormap defined by nanonisMap (NanoLib/op/nanonisMap.m).\\
\textbf{Note:}The x and y axis are in pixels and the origin is top/left corner(matrix representation).
\item Lets the user draw the line and drag or modify it after created.
\item Plots the line profile and upload it every time the drawn line is moved or modified.
\textbf{Note:} The intensity of the line is not explicit but it has the same units of the data.
\end{itemize}

\subsection{Code2GetResolution}

This script opens a sxm file and applies \textbf{GetResolution} function to a topography map(Z channel).

\subsubsection{Function GetResolution}

This function acts as following:
\begin{itemize}  
\item Plots alldata matrix using the colormap defined by nanonisMap (NanoLib/op/nanonisMap.m).\\
\textbf{Note:}The $x$ and $y$ axis are in pixels and the origin is top/left corner(matrix representation).
\item Lets the user load two text files which are supposed to contain two set of points inside the image representing the borders the user wants to compare.
\item Plots the borders from the files.
\item The code starts to compare for every line the $x$ position of the of the first border with the $x'$ position of the second border, creating the histogram called ''reshist'' of the displacement. Those scan line that do not contain points of one of the two borders are excluded.
\item When one of the two borders is dragged the histogram is updated.
\item The total shift between the two borders and the standard deviation is shown in the title of the reshist figure.
\item Gives to the user the possibility to rotate one of the borders respect to a point chosen using the button present on top of the Data figure. 
\item A default threshold of 35 pixels is used to cut out those scan lines where the distance between the points of the different borders was unreasonable big(more then 35 pxs), due to a possible failure of the step border method used to find the border in the nfesem image.
\end{itemize}

%%\noindent Other examples can be found in the section \ref{sec::examples}.
